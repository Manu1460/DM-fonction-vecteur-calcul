\documentclass[12pt,a4paper]{article}
\usepackage[utf8]{inputenc}
\usepackage[T1]{fontenc}
\usepackage{amsmath}
\usepackage{amsfonts}
\usepackage{amssymb}
\usepackage{multicol}
\usepackage{qrcode}
\usepackage{lmodern}
\usepackage{colortbl}%permet de griser les cases
\usepackage{tabularx, multirow}
%\usepackage{lscape}
\usepackage{xcolor}
%\usepackage{graphicx}
\usepackage{tikz,tkz-base}
\input{preambulemanu.sty}
\usepackage[left=2cm,right=2cm,top=2cm,bottom=2cm]{geometry}
\def\Oij{$\left(\text{O},~\vec{i},~\vec{j}\right)$}
\usepackage{fancyhdr}
\usepackage{MnSymbol,wasysym}

%Permet le code python sur lateX
\usepackage{minted}
\usemintedstyle{lovelace}
\begin{document}


\textbf{2nd} \hfill \textbf{DM bilan du premier trimestre} \hfill Lycée Jean Rostand\\
\trait 
\subsection*{Exercice 1 \hfill 5pts}
Voici la courbe $C_f$ d'une fonction $f$. Répondez graphiquement aux questions suivantes:

\begin{center}
\includegraphics[scale=0.7
]{courbe.png}
\end{center}


\begin{enumerate}
\item Quel est le domaine de définition de la fonction $f$ ?
\item Quelle est l' image de 2 ? Quelle est la valeur de $f(-6)$  ?
\item Quels sont les antécédents de $6$ ?
\item Donner les solutions de $f(x)=-4$.
\item Donner les solutions de l'inéquation $f(x)>4$.
\item Préciser les solutions de $f(x)=0$ .Établir le tableau de signes de la fonction $f$.
\item Établir le tableau de variations de $f$.
\item Quel est le maximum de $f$ sur $[0;9]$ ? Quel est le minimum de $f$ sur $[-8;0]$ ?


\end{enumerate}

\subsection*{Exercice 2 \hfill 6pts}
On considère 3 points A,B et C tels que AB= 6cm, AC=8cm et BC =10cm.

\begin{enumerate}
\item Tracer le triangle ABC au compas.
\item Prouver que ABC est un triangle rectangle
\item Construire N le symétrique du point A par rapport au centre B.
\item Construire  M l'image du point C par la translation de vecteur $\overrightarrow{BA}$
\item Construire le point R tel que $\overrightarrow{BR}=\overrightarrow{BN}+\overrightarrow{BC}$
\item Construire le point T tel que $\overrightarrow{RT}=\overrightarrow{AC}+\overrightarrow{CB}$
\item Prouver que le quadrilatère NBRT est un rectangle. 
\end{enumerate}
 

 


\subsection*{Exercice 3 \hfill 5pts}

On considère la fonction  sur $[-4;5]$ par $f(x)=1,5x^3-2x+4$. \\
On note $C_f$ sa courbe dans un repère du plan. 
\begin{enumerate}
\item Calculer l'image de -2 par la fonction $f$ en détaillant les calculs.
\item 
Compléter l'algorithme en langage python qui permet de calculer les images de la fonction.

 \begin{minted}[gobble=1]{python} 
 def fonction(x):
     y=.........
     return(....)
\end{minted}


\item En utilisant la table de votre calculatrice compléter le tableau suivant:

\begin{center}
    

\renewcommand{\arraystretch}{2}

\begin{tabular}{|>{\centering}p{0.7cm}|>{\centering}p{0.6cm}|>{\centering}p{0.6cm}|>{\centering}p{0.6cm}|>{\centering}p{0.6cm}|>{\centering}p{0.6cm}|>{\centering}p{0.6cm}|>{\centering}p{0.6cm}|>{\centering}p{0.6cm}|>{\centering}p{0.6cm}|>{\centering}p{0.6cm}|}
\hline
$x$  & -4 & -3 & -2& -1 & 0 & 1 & 2 & 3& 4 & 5 \tabularnewline\hline 
$f(x)$ &  &  &  & &  &  &  &  & &  \tabularnewline 

\hline 

\end{tabular} 
\end{center}

\item Faites apparaître la courbe $C_f$ sur votre calculatrice sur $[-4;5]$.
Puis représenter votre courbe dans un repère adapté sur votre copie pour $-4\leq x\leq5$.
\item Etudier la parité de la fonction.
\end{enumerate} 

\subsection*{Exercice 4 \hfill 4pts}

\begin{enumerate}
 
\item Simplifier en détaillant $A=10^4\times \dfrac{10^{-2}}{\left(10^{-7}\right)^5}$.
\item Calculer et donner la forme irréductible de : $\dfrac{1+\dfrac{5}{6}}{2-\dfrac{2}{3}}$
\item En donnant toutes les étapes, trouver le résultat irréductible de $B=\dfrac{5}{3}-\dfrac{1}{5}+1 $.
\item Résoudre l'équation: $8x-17=-3x-24 $.





\end{enumerate}





\end{document}
